\documentclass[a4paper,11pt,titlepage]{scrreprt} 

% Eingebundene Unterstützungspakete
\usepackage[T1]{fontenc}            % deutsche Umlaute und ß
%\usepackage[latin1]{inputenc}      %unter Windows
\usepackage[utf8]{inputenc}         %unter Linux
\usepackage{ngerman}                 % deutsche Schlüsselwörter
\usepackage{amsmath,amsfonts, amssymb, amsthm, mathrsfs} % mathematische Zeichen
\usepackage{bbm}                    % Doppelstrichsymbole
\usepackage{graphicx}               % Einbinden von Graphiken
\usepackage{textcomp,doc}
\usepackage{cmbright}               % Fontauswahl, vgl. http://www.tug.dk/FontCatalogue/
\usepackage{courier}                % für Typewriter \tt


% Festlegung von Dokumentgrößen
\parskip1ex                         % Abstand zwischen zwei Abschnitten
\parindent0mm                       % keine Einrückung bei neuem Abschnitt

% Festlegung der Numerierungen
\newtheorem{definition}{Definition}[chapter]
\newtheorem{example}[definition]{Beispiel}
\newtheorem{theorem}[definition] {Theorem}
\newtheorem{algorithm}{Algorithmus}

% Trennungsregeln für bestimmte Wörter
\hyphenation{Po-ly-nom-in-ter-po-la-ti-on}
\hyphenation{Form-op-ti-mie-rung}

% Definition eigener Befehle und Macros
\newcommand{\ACS}{2pt}              % Matrizen und Vektoren
\newcommand{\mat}[4]{{\arraycolsep\ACS
\left#1\begin{array}{@{}*{#2}{c}@{}}#4\end{array}\right#3}}
\newcommand{\R}{{\mathbb{R}}}       % reelle Zahlen
\newcommand{\LL}{{\mathscr{L}}}%%{{\mathcal{L}}}




%%%%%%%%%%%%%%%%%%%%%%%%%%%%%%%%%%%%%%%%%%%%%%%%%%%%%%%%%%%%%%%%%%%%%%%%%%%%%%%%%%%%%%%%
\begin{document}
%%%%%%%%%%%%%%%%%%%%%%%%%%%%%%%%%%%%%%%%%%%%%%%%%%%%%%%%%%%%%%%%%%%%%%%%%%%%%%%%%%%%%%%%

\begin{titlepage}
    \begin{center}
    \huge \textbf{\textsf{Szenariobasierte Generierung von Testfällen im European~Train~Control~System}} \\
    \vspace{2cm}
    \LARGE\textbf{\textsc{Bachelor-Thesis}}\\
    \vspace{1cm}
    \normalsize
    vorgelegt am: \today \\
    \vspace{2.5cm}
    \large \textbf{am Fachbereich~Berufsakademie der Fachhochschule~für~Wirtschaft~Berlin}\\
    \vspace{3cm}
    \end{center}
 \normalsize{
    \begin{tabular}{ll}
    	Name: & {dein Name} \\
    	Matrikelnummer: & {die Mtrnr.} \\
    	Ausbildungsbetrieb: & Alcatel-Lucent Deutschland AG\\
    	Fachbereich: & Berufsakademie\\
    	Studienbereich: & Technik\\
    	Studiengang: & Informatik\\
    	Studienjahrgang: & 2004\\
      Erstgutachter: & {der erste} \\
      Zweitgutachter: & {der zweite} \\
    \end{tabular}\\
    }
\end{titlepage}

\pagenumbering{roman}

\begin{center}
\large
\phantom{erste Zeile}
\vspace{2.5cm}
\textbf{Erklärung zur Diplomarbeit}\\
\vspace{0.5cm}
\end{center}
\normalsize
Hiermit erkläre ich, daß ich die Diplomarbeit selbständig verfasst und keine anderen 
als die angegebenen Quellen und Hilfsmittel benutzt und die aus fremden Quellen direkt 
oder indirekt übernommenen Gedanken als solche kenntlich gemacht habe. Die Diplomarbeit 
habe ich bisher keinem anderen Prüfungsamt in gleicher oder vergleichbarer Form vorgelegt. 
Sie wurde bisher auch nicht veröffentlicht.

\vspace{9cm}

\today\\              % oder eigenes Datum

\smallskip
\small\hspace{0cm}(Datum) \hspace{8cm} (Unterschrift)
\normalsize
\newpage

\tableofcontents      % Inhaltsverzeichnis
%\listoffigures        % Abbildungsverzeichnis

\newpage
\setcounter{page}{1}
\pagenumbering{arabic}

Hello

\end{document}