%%%%%%%%%%%%%%%%%%%%%%%%%%%%%%%%%%%%%%%%%%%%%%%%%%%%%%%%%%%%%%%%%%%%%%%%%%%%%%%%%%%%%%%%
%
% Masterthesis in DataScience
%
%%%%%%%%%%%%%%%%%%%%%%%%%%%%%%%%%%%%%%%%%%%%%%%%%%%%%%%%%%%%%%%%%%%%%%%%%%%%%%%%%%%%%%%%
% Sean Kallenberg, University of Trier, 2020
%%%%%%%%%%%%%%%%%%%%%%%%%%%%%%%%%%%%%%%%%%%%%%%%%%%%%%%%%%%%%%%%%%%%%%%%%%%%%%%%%%%%%%%%

\documentclass[a4paper,12pt,titlepage,enabledeprecatedfontcommands]{scrreprt} 

\usepackage[T1]{fontenc}            
\usepackage[utf8]{inputenc}         
\usepackage{amsmath,amsfonts, amssymb, amsthm, mathrsfs}
\usepackage{bbm}                    
\usepackage{graphicx}               
\usepackage{textcomp,doc}
\usepackage{cmbright}               
\usepackage{courier}                
\usepackage{color}							
\usepackage{exscale}						
\usepackage[english]{babel}		
\usepackage{times}						
\usepackage{graphicx}						
\usepackage{pgf}	
\usepackage{listings}
\usepackage{cite}
\usepackage[ruled,vlined]{algorithm2e}
\usepackage{bm}
\usepackage{url}
\usepackage{hyperref}
\usepackage{wrapfig}
\usepackage{subfig}

\parskip1ex                         
\parindent0mm                       

\newtheorem{definition}{Definition}[chapter]
\newtheorem{example}[definition]{Beispiel}
\newtheorem{theorem}[definition]{Theorem}

\begin{document}


\begin{titlepage}
\phantom{erste Zeile}
\begin{center}
\Huge
\textbf{Multigrid Methods \\ in \\Neuronal Networks}\\
\vspace{0.5cm}
\large
\vspace{3cm}
MASTER THESIS \\
\vspace{0.5cm}
to obtain the academic degree of \\
Master of Science \\
\vspace{1cm}
submitted to the Department IV - Data Science\\
of the University of Trier \\
\vspace{2.5cm}
by \\
\vspace{1cm}
Sean, Kallenberg,\\
Zurmaienerstraße 166 \\
54292 Trier \\
\vspace{1cm}
Supervisor: Prof. Dr. Volker Schulz \\
\vspace{1cm}
Trier, \today         
\end{center}
\normalsize
\vfill
\end{titlepage}

\pagenumbering{roman}
\large
\vspace{2.5cm}
\begin{center}
\textbf{Statutory Declaration}\\
\end{center}
\vspace{0.5cm}
\normalsize
I hereby declare that I wrote this thesis independently and have not used any other source or tool than the ones indicated. Any thought taken directly or indirectly from external sources are marked as such. This master thesis was not presented to any other examination office in the same or similar form. It was not yet published.

\vspace{9cm}

\today\\             

\smallskip
\small\hspace{0cm}(Date) \hspace{8cm} (Signature)
\normalsize
\newpage

\tableofcontents
\listoffigures

\newpage
\setcounter{page}{1}
\pagenumbering{arabic}

\chapter{Introduction}
\section{Historic Background}
The task of image analysis has a steadily growing importance in everyday situations. Whether it is in form of classification, segmentation or object detection - many modern applications rely on the nowadays reachable results. This is mostly thanks due to the astounding precision the research community was able to produce using artificial neuronal networks (ANN). The idea to base computational models on the neuronal system of the human brain was first introduced in the 1940s by Warren McCulloch and Warren Pitts \cite{10.5555/65669.104377} and quickly further build upon by Donald Hebb \cite{hebb-organization-of-behavior-1949}. But soon research reached its first bottleneck by the limited computational resources at the time as stated Marvin Minsky and Seymour Papert \cite{minsky69perceptrons} by. It was not possible to build and train large networks that would produce usable results. Thus the idea of a learned model to classify input was replaced through systems based on if-then decisions. Even though Paul John Werbos introduced the to this day used training algorithm of backprogagation in 1975 \cite{werbos1975beyond} it took until the new millenia and the invention of high performing GPUs to make superhuman precision through ANNs possible.
\chapter{Related Work}
\chapter{Multigrid Methods}
\section{Fundamentals}
\section{Iterative Algorithms}
\section{Convergence Theory}
\chapter{Artificial Neuronal Networks}
\section{Concept}
\section{Convolutional Networks}
\section{Sophisticated Network Architectures}
\chapter{Application of Multigrid Methods in CNNs}
\section{Multigrid as Network Architecture}
\section{Multigrid as Forward Propagation}
\chapter{Implementation and empirical Analysis}
\section{Chosen Network Architectures}
\section{Realization in Python}
\section{Analysis}
\chapter{Conclusion}
\cite{Goodfellow-et-al-2016}

\addcontentsline{toc}{chapter}{Bibliography}
\bibliographystyle{plain}
\bibliography{references} \label{bibtex}
\end{document}